\begin{itemize}
\item \label{sec:sbol:displayId} 
The \sbol{displayId} property is an OPTIONAL identifier with a data type of \paml{string} (and REQUIRED for objects with URL identifiers). This property is intended to be an intermediate between a URI and the \sbol{name} property that is machine-readable, but more human-readable than the full URI of an object.
If set, its \paml{string} value MUST be composed of only alphanumeric or underscore characters and MUST NOT begin with a digit.

\item \label{sec:sbol:name}
The \sbol{name} property is OPTIONAL and has a data type of \paml{string}. This property is intended to be displayed to a human when visualizing an \sbol{Identified} object.
If an \sbol{Identified} object lacks a name, then software tools SHOULD instead display the object's \sbol{displayId} or URI.

\item \label{sec:sbol:description}
The \sbol{description} property is OPTIONAL and has a data type of \paml{string}. This property is intended to contain a more thorough text description of an \sbol{Identified} object.

\item \label{sec:prov:wasDerivedFrom}
The \prov{wasDerivedFrom} property MAY contain any number of \paml{URI}s. This property is defined by the PROV-O ontology and is located in the \url{https://www.w3.org/ns/prov#} namespace.

\item \label{sec:prov:wasGeneratedBy}
The \prov{wasGeneratedBy} property MAY contain any number of \paml{URI}s. This property is defined by the PROV-O ontology and is located in the \url{https://www.w3.org/ns/prov#} namespace.

\item \label{sec:sbol:hasMeasure}
The \sbol{hasMeasure} property MAY contain any number of \paml{URI}s, each of which refers to a \om{Measure} object that describes a measured parameter for this object.
\end{itemize}

\subsubsection{sbol:TopLevel}
\label{sec:sbol:TopLevel}

\sbol{TopLevel} is an abstract class that is extended by any \sbol{Identified} class that can be found at the top level of a PAML or SBOL document or file.
In other words, \sbol{TopLevel} objects are never nested inside of any other object as a child object.
The \sbol{TopLevel} classes defined in PAML are \paml{Protocol} and \paml{Primitive}. 

\begin{figure}[ht]
\begin{center}
\includegraphics[scale=0.6]{sbol_classes/toplevel}
\caption[]{Classes that inherit from the \sbol{TopLevel} abstract class.}
\label{uml:toplevel}
\end{center}
\end{figure}

\begin{itemize}
\item \label{sec:sbol:hasNamespace}
The \sbol{hasNamespace} property is REQUIRED and MUST contain a single \paml{URI} that defines the namespace portion of URLs for this object and any child objects.
If the URI for the \sbol{TopLevel} object is a URL, then the URI of the \sbol{hasNamespace} property MUST prefix match that URL.

\item 
\label{sec:sbol:hasAttachment}
The \sbol{hasAttachment} property MAY have any number of \paml{URI}s, each referring to an \external{sbol:Attachment} object.
\end{itemize}


\subsubsection{sbol:Component}
\label{sec:sbol:Component}

The \sbol{Component} class represents the structural and/or functional entities of a biological design. 
In PAML, this is primarily used to represent the design of experimental samples as combinations of entities such as strains, genetic constructs, media, inducers, etc.

As shown in \ref{uml:component}, the \sbol{Component} class describes a design entity using a number of different properties.
In many PAML usages, however, a \sbol{Component} will simply be used as a pointer to an external description of a material to be manipulated, and the only property required for interpreting PAML will be \sbolmult{type:C}{type}.

\begin{figure}[ht]
\begin{center}
\includegraphics[scale=0.6]{sbol_classes/component}
\caption[]{Diagram of the \sbol{Component} class and its associated properties.}
\label{uml:component}
\end{center}
\end{figure} 

\begin{itemize}
\item \label{sec:sbol:type:C}
The \sbolmult{type:C}{type} property MUST have one or more \paml{URI}s specifying the category of biochemical or physical entity (for example DNA, protein, or simple chemical) that a \sbol{Component} object represents.

\item \label{sec:sbol:hasFeature}
The \sbol{hasFeature} property MAY have any number of \paml{URI}s, each referencing a \sbol{Feature} object. Each \sbol{Feature} represents a specific occurrence of a part, subsystem, or other notable aspect within that design, such as an ingredient in the composition of a growth medium.
\\{\em This is not typically required for specifying protocols in PAML.}

\item \label{sec:sbol:role:C}
The \sbolmult{role:C}{role} property MAY have any number of \paml{URI}s, which MUST identify terms from ontologies that are consistent with the \sbolmult{type:C}{type} property of the \sbol{Component}. 
\\{\em This is not typically required for specifying protocols in PAML.}

\item \label{sec:sbol:hasSequence:C}
The \sbolmult{hasSequence:C}{hasSequence} property MAY have any number of \paml{URI}s, each referencing a \external{sbol:Sequence} object.  These objects define the primary structure or structures of the \sbol{Component}.
\\{\em This is not typically required for specifying protocols in PAML.}

\item \label{sec:sbol:hasConstraint}
The \sbol{hasConstraint} property MAY have any number of \paml{URI}s, each referencing a \external{sbol:Constraint} object.
These objects describe, among other things, any restrictions on the relative, sequence-based positions and/or orientations of the \sbol{Feature} objects contained by the \sbol{Component}, as well as spatial relations such as containment and identity relations.
\\{\em This is not typically required for specifying protocols in PAML.}

\item \label{sec:sbol:hasInteraction}
The \sbol{hasInteraction} property MAY have any number of \paml{URI}s, each referencing an \external{sbol:Interaction} object describing a behavioral relationship between \sbol{Feature}s in the \sbol{Component}.
\\{\em This is not typically required for specifying protocols in PAML.}

\item \label{sec:sbol:hasInterface}
The \sbol{hasInterface} property is OPTIONAL and MAY have a \paml{URI} referencing an \external{sbol:Interface} object that indicates the inputs, outputs, and non-directional points of connection to a \sbol{Component}.
\\{\em This is not typically required for specifying protocols in PAML.}

\item \label{sec:sbol:hasModel}
The \sbol{hasModel} property MAY have any number of \paml{URI}s, each referencing a \external{sbol:Model} object that links the \sbol{Component} to a computational model in any format.
\\{\em This is not typically required for specifying protocols in PAML.}
\end{itemize}


\subsection{Ontology of Units of Measure}

In most cases where a number is needed in PAML, that number is a measure with units associated with it.
The Ontology of Units of Measure (OM) (\url{http://www.ontology-of-units-of-measure.org/resource/om-2}) already defines a data model for representing measures and their associated units. 
A subset of OM, already adopted by SBOL, is used for this purpose by PAML as well.

The key class used is \om{Measure}, which associates a number with a unit and a biology-related property.
In most cases, it should be possible to use one of the \external{om:Unit} instances already defined by OM; when this is not possible, an appropriate unit can be defined using \external{om:Unit} and \external{om:Prefix} classes.

\subsubsection{om:Measure} \label{sec:om:Measure}

The purpose of the \om{Measure} class is to link a numerical value to a \external{om:Unit}. 

\begin{itemize}
\item \label{sec:om:hasNumericalValue}
The \om{hasNumericalValue} property is REQUIRED and MUST contain a single \paml{float}.

\item \label{sec:om:hasUnit:Measure}
The \ommult{hasUnit:Measure}{hasUnit} property is REQUIRED and MUST contain a \paml{URI} that refers to a \external{om:Unit}. 

\item \label{sec:sbol:type:Measure}
The \sbolmult{type:Measure}{type} property MAY contain any number of \paml{URI}s. It is RECOMMENDED that one of these \paml{URI}s identify a term from the Systems Description Parameter branch of the Systems Biology Ontology (SBO) (\url{http://www.ebi.ac.uk/sbo/main/}). This \sbolmult{type:Measure}{type} property was added by SBOL to describe different types of parameters 
(for example, rate of reaction is identified by the SBO term \url{http://identifiers.org/SBO:0000612}).
\end{itemize}

\subsection{Recommended Ontologies for External Terms}
\label{sec:recomm_ontologies}

External ontologies and controlled vocabularies are an integral part of SBOL and thus used by PAML as well. SBOL uses \paml{URI}s to access existing biological information through these resources. 
Although RECOMMENDED ontologies have been indicated in relevant sections where possible, other resources providing similar terms can also be used. A summary of these external sources can be found in \ref{tbl:preferred_external_resources}.
The URIs for ontological terms SHOULD come from \url{identifiers.org}.  However, it is acceptable to use terms from \url{purl.org} as an alternative, for example when RDF tooling requires URIs to be represented as compliant QNames, and software may convert between these forms as required.

\begin{table}[htp]
  \begin{edtable}{tabular}{p{2cm}p{1.5cm}p{5cm}p{6cm}}
    \toprule
    \textbf{SBOL Entity} & \textbf{Property} & \textbf{Preferred External Resource}
    & \textbf{More Information} \\
    \midrule
    \textbf{Component}  & type & SBO (physical entity branch)& \url{http://www.ebi.ac.uk/sbo/main/}\\
                                  & type & SO (nucleic acid topology)& \url{http://www.sequenceontology.org}\\
    						   	  & role & SO (\textit{DNA} or \textit{RNA}) & \url{http://www.sequenceontology.org}   \\
    						   	  & role & CHEBI (\textit{small molecule}) & \url{https://www.ebi.ac.uk/chebi/}   \\
							  & role & PubChem (\textit{small molecule}) & \url{https://pubchem.ncbi.nlm.nih.gov/} \\
    						   	  & role & UniProt (\textit{protein}) & \url{https://www.uniprot.org/}  \\   
    						   	  & role & NCIT (\textit{samples}) & \url{https://ncithesaurus.nci.nih.gov/}  \\   
    \textbf{om:Measure}	& type & SBO (systems description parameters) &
    \url{http://www.ebi.ac.uk/sbo/main/} \\
    \bottomrule
  \end{edtable}
  \caption{Preferred external resources from which to draw values for various SBOL properties.}
  \label{tbl:preferred_external_resources}
\end{table}

